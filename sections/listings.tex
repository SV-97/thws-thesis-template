\chapter{Listings}

\section{Recursive algorithm for finding the pointwise minimum of a set of affine functions}
\label{sec:pointwise-min-rec-py}

We start out with a few imports and the type definitions we need.

\inputminted[firstline=1, lastline=25]{python}{listings/pointwise_min_rec.py}

Afterwards we'll define the top level function taking a list of affine functions and optionally (to account for the case where the input is empty) returning a piecewise affine function representing the piecewise minimum. Note that this uses a not yet defined helper that will manage the actually recursive core.

\inputminted[firstline=28, lastline=48]{python}{listings/pointwise_min_rec.py}

Lastly we define the core function that (in this case lazily) constructs the leaves of the tree.

\inputminted[firstline=51, lastline=79]{python}{listings/pointwise_min_rec.py}

A little example might look like this.

\inputminted[firstline=83, lastline=84]{python}{listings/pointwise_min_rec.py}