\usepackage{pdfpages}
\usepackage[english]{babel}
\usepackage[hyphens]{url} % break long url's with hyphens
\usepackage{hyperref}
\usepackage[utf8]{inputenc}
% \usepackage[a4paper, left=30mm, right=30mm, top=30mm, bottom=40mm]{geometry}
\usepackage[a4paper]{geometry}
\newgeometry{twoside}

% colors
\usepackage{xcolor}
% some of the colours from tableau 10
% https://help.tableau.com/current/pro/desktop/en-us/formatting_create_custom_colors.htm
\definecolor{mpl-blue}{HTML}{1f77b4}
\definecolor{mpl-orange}{HTML}{fba402}
\definecolor{mpl-green}{HTML}{2ca02c}
\definecolor{mpl-red}{HTML}{d62728}
\definecolor{mpl-purple}{HTML}{9467bd}
\definecolor{mpl-brown}{HTML}{8c564b}
\definecolor{mpl-pink}{HTML}{e377c2}
\definecolor{mpl-grey}{HTML}{7f7f7f}
\definecolor{mpl-gray}{HTML}{7f7f7f}
\definecolor{mpl-lime}{HTML}{bcbd22}
\definecolor{mpl-cyan}{HTML}{17becf}
\definecolor{links-color}{HTML}{044f83} % 020f54

\hypersetup{
    % allbordercolors={0 0 0},
    pdfborderstyle={/S/U/W 0.1},
    % this sets some pdf metadata
    pdftitle={Title of the thesis},
    pdfauthor={Author Name},
    colorlinks=true,
    allcolors=links-color,
}

%\usepackage{multicol}
\usepackage[breakable]{tcolorbox}
\usepackage{csquotes}

\usepackage{caption}
\usepackage{subcaption}

\usepackage{float}

\usepackage{footmisc} % footnote references

\usepackage{amsmath}
\usepackage{amssymb}
\usepackage{amsfonts}
\usepackage{amsthm}
\usepackage{mathtools}

% for mathscr
\usepackage{mathrsfs}
%\usepackage{newtxmath}

% commutative diagrams stuff
\usepackage{quiver}
\usepackage{tikz}
\usetikzlibrary{cd}
\usetikzlibrary{babel}

\usepackage{enumerate}
\usepackage[shortlabels]{enumitem}

\usepackage{algorithm2e}

% for diagonally split cells in tables
\usepackage{diagbox}

% manage fonts
\usepackage{fontspec}
% alternate monospace font - has to be installed
% \setmonofont{Fira Code}[Scale=MatchLowercase]


% for listings
\usepackage{minted}
\setminted[python]{
    %fontsize=\footnotesize,
    frame=lines,
    linenos,
    samepage,
}


\usepackage[style=alphabetic]{biblatex}
\addbibresource{sources.bib}


\usepackage{fancyhdr}

\pagestyle{fancy} %eigener Seitenstil
\fancyhf{} %alle Kopf- und Fußzeilenfelder bereinigen
% Mit \rightmark geht Subsection Name

\fancyhead[LO,RE]{\rightmark} %Kopfzeile links
\fancyhead[RO,LE]{\thepage} %Kopfzeile rechts
\renewcommand{\headrulewidth}{0.4pt} %obere Trennlinie
% \fancyfoot[C]{\thepage} %Seitennummer
%\renewcommand{\footrulewidth}{0.4pt} %untere Trennlinie

% for odd / even page conditional
\usepackage[strict]{changepage}

% set special style for "plain" pages (chapter start, bibliography etc.)
\fancypagestyle{plain}{
    \fancyhf{} % clear header / footer style
    \fancyfoot[RO,LE]{\thepage} % page number in bottom right corner
    \renewcommand{\headrulewidth}{0.0pt} %obere Trennlinie
    % change lower bar to thick bar in the right corner
    \renewcommand{\footrule}{
        \checkoddpage
        \ifoddpage
            \makebox[\headwidth][r]{\rule{.075\headwidth}{0.1cm}}
        \else
            \makebox[\headwidth][l]{\rule{.075\headwidth}{0.1cm}}
        \fi
    }
}

\fancypagestyle{empty}{
    \fancyhf{} % clear header / footer style
    \renewcommand{\headrulewidth}{0.0pt} %obere Trennlinie
}

\setlength{\headheight}{14pt}

% start chapters on even pages
% \makeatletter
% \renewcommand*\cleardoublepage{\clearpage\if@twoside
%         \ifodd\c@page \hbox{}\newpage\if@twocolumn\hbox{}%
%                 \newpage\fi\fi\fi}
% \makeatother

% table of contents
% nottoc to hide "Contents" from TOC
\usepackage[nottoc]{tocbibind}
\setcounter{tocdepth}{2}
\setlength{\parindent}{0pt}

% math stuff

% custom theorem style für newlines zu beginn des Satzes
\newtheoremstyle{mainTheoremStyle}%    <name>
{\topsep}%   <space above>
{\topsep}%   <space below>
{}%  <body font> % was \itshape
{}%          <indent amount>
{\bfseries}% <Theorem head font>
{.}%         <punctuation after theorem head>
{\newline}%  <space after theorem head> (default .5em)
{}%          <Theorem head spec>
\theoremstyle{mainTheoremStyle}

% internal / raw theorem environments
\newtheorem{int_theorem}{Theorem}[section]
\newtheorem{int_definition}[int_theorem]{Definition}
\newtheorem{notation}[int_theorem]{Notation}
\newtheorem{int_lemma}[int_theorem]{Lemma}
\newtheorem{int_corollary}[int_theorem]{Corolloary}
\newtheorem{int_remark}[int_theorem]{Remark}
\newtheorem{int_example}[int_theorem]{Example}

\newenvironment{theorem}[1][]{%        % Create new environment which wraps our Theorem into a tcolorbox.
\begin{tcolorbox}[
    breakable,
    colback=blue!5!white,%     Background color.
    width=\dimexpr\linewidth+10pt\relax,%     Allow your box to be bigger than \linewidth ...
    enlarge left by=-5pt,%                    ... in order to have the text properly aligned. ...
    enlarge right by=-5pt,%                   ... Note that boxsep = -enlargeLeft = -enlargeRight = 0.5*enlargement of width. ...
    boxsep=5pt,%                              ... This is necessary to keep everything good looking.
    left=0pt,%                                Avoid extra space on the left, ...
    right=0pt,%                               ... right, ...
    top=0pt,%                                 ... top, ...
    bottom=0pt,%                              ... and bottom.
    arc=0pt,%                                 Corners not rounded.
    boxrule=0pt,%                             No boxrule.
    colframe=white]{}{}%                      Make rest of the boxrule invisible.
\ifstrempty{#1}{%                         If you didn't specify the optional argument of Theorem ...
    \begin{int_theorem}%                     ... then open a normal Theorem ...

}{%                                       ... else ...
    \begin{int_theorem}[#1]%                 ... open a Theorem and use the optional argument.
        }%
        }{%
    \end{int_theorem}%                          Close every environment.
    \end{tcolorbox}%
}

\newenvironment{definition}[1][]{%        % Create new environment which wraps our Theorem into a tcolorbox.
\begin{tcolorbox}[
    breakable,
    colback=red!5!white,%     Background color.
    width=\dimexpr\linewidth+10pt\relax,%     Allow your box to be bigger than \linewidth ...
    enlarge left by=-5pt,%                    ... in order to have the text properly aligned. ...
    enlarge right by=-5pt,%                   ... Note that boxsep = -enlargeLeft = -enlargeRight = 0.5*enlargement of width. ...
    boxsep=5pt,%                              ... This is necessary to keep everything good looking.
    left=0pt,%                                Avoid extra space on the left, ...
    right=0pt,%                               ... right, ...
    top=0pt,%                                 ... top, ...
    bottom=0pt,%                              ... and bottom.
    arc=0pt,%                                 Corners not rounded.
    boxrule=0pt,%                             No boxrule.
    colframe=white]{}{}%                      Make rest of the boxrule invisible.
\ifstrempty{#1}{%                         If you didn't specify the optional argument of Theorem ...
    \begin{int_definition}%                     ... then open a normal Theorem ...
}{%                                       ... else ...
    \begin{int_definition}[#1]%                 ... open a Theorem and use the optional argument.
        }%
        }{%
    \end{int_definition}%                          Close every environment.
    \end{tcolorbox}%
}

\newenvironment{lemma}[1][]{%        % Create new environment which wraps our Theorem into a tcolorbox.
\begin{tcolorbox}[
    breakable,
    colback=yellow!5!white,%     Background color.
    width=\dimexpr\linewidth+10pt\relax,%     Allow your box to be bigger than \linewidth ...
    enlarge left by=-5pt,%                    ... in order to have the text properly aligned. ...
    enlarge right by=-5pt,%                   ... Note that boxsep = -enlargeLeft = -enlargeRight = 0.5*enlargement of width. ...
    boxsep=5pt,%                              ... This is necessary to keep everything good looking.
    left=0pt,%                                Avoid extra space on the left, ...
    right=0pt,%                               ... right, ...
    top=0pt,%                                 ... top, ...
    bottom=0pt,%                              ... and bottom.
    arc=0pt,%                                 Corners not rounded.
    boxrule=0pt,%                             No boxrule.
    colframe=white]{}{}%                      Make rest of the boxrule invisible.
\ifstrempty{#1}{%                         If you didn't specify the optional argument of Theorem ...
    \begin{int_lemma}%                     ... then open a normal Theorem ...
}{%                                       ... else ...
    \begin{int_lemma}[#1]%                 ... open a Theorem and use the optional argument.
        }%
        }{%
    \end{int_lemma}%                          Close every environment.
    \end{tcolorbox}%
}

\newenvironment{corollary}[1][]{%        % Create new environment which wraps our Theorem into a tcolorbox.
\begin{tcolorbox}[
    breakable,
    colback=purple!5!white,%     Background color.
    width=\dimexpr\linewidth+10pt\relax,%     Allow your box to be bigger than \linewidth ...
    enlarge left by=-5pt,%                    ... in order to have the text properly aligned. ...
    enlarge right by=-5pt,%                   ... Note that boxsep = -enlargeLeft = -enlargeRight = 0.5*enlargement of width. ...
    boxsep=5pt,%                              ... This is necessary to keep everything good looking.
    left=0pt,%                                Avoid extra space on the left, ...
    right=0pt,%                               ... right, ...
    top=0pt,%                                 ... top, ...
    bottom=0pt,%                              ... and bottom.
    arc=0pt,%                                 Corners not rounded.
    boxrule=0pt,%                             No boxrule.
    colframe=white]{}{}%                      Make rest of the boxrule invisible.
\ifstrempty{#1}{%                         If you didn't specify the optional argument of Theorem ...
    \begin{int_corollary}%                     ... then open a normal Theorem ...
}{%                                       ... else ...
    \begin{int_corollary}[#1]%                 ... open a Theorem and use the optional argument.
        }%
        }{%
    \end{int_corollary}%                          Close every environment.
    \end{tcolorbox}%
}

\newenvironment{remark}[1][]{%        % Create new environment which wraps our Theorem into a tcolorbox.
\begin{tcolorbox}[
    breakable,
    colback=green!5!white,%     Background color.
    width=\dimexpr\linewidth+10pt\relax,%     Allow your box to be bigger than \linewidth ...
    enlarge left by=-5pt,%                    ... in order to have the text properly aligned. ...
    enlarge right by=-5pt,%                   ... Note that boxsep = -enlargeLeft = -enlargeRight = 0.5*enlargement of width. ...
    boxsep=5pt,%                              ... This is necessary to keep everything good looking.
    left=0pt,%                                Avoid extra space on the left, ...
    right=0pt,%                               ... right, ...
    top=0pt,%                                 ... top, ...
    bottom=0pt,%                              ... and bottom.
    arc=0pt,%                                 Corners not rounded.
    boxrule=0pt,%                             No boxrule.
    colframe=white]{}{}%                      Make rest of the boxrule invisible.
\ifstrempty{#1}{%                         If you didn't specify the optional argument of Theorem ...
    \begin{int_remark}%                     ... then open a normal Theorem ...
}{%                                       ... else ...
    \begin{int_remark}[#1]%                 ... open a Theorem and use the optional argument.
        }%
        }{%
    \end{int_remark}%                          Close every environment.
    \end{tcolorbox}%
}

\newenvironment{example}[1][]{%        % Create new environment which wraps our Theorem into a tcolorbox.
\begin{tcolorbox}[
    breakable,
    colback=yellow!5!white,%     Background color.
    width=\dimexpr\linewidth+10pt\relax,%     Allow your box to be bigger than \linewidth ...
    enlarge left by=-5pt,%                    ... in order to have the text properly aligned. ...
    enlarge right by=-5pt,%                   ... Note that boxsep = -enlargeLeft = -enlargeRight = 0.5*enlargement of width. ...
    boxsep=5pt,%                              ... This is necessary to keep everything good looking.
    left=0pt,%                                Avoid extra space on the left, ...
    right=0pt,%                               ... right, ...
    top=0pt,%                                 ... top, ...
    bottom=0pt,%                              ... and bottom.
    arc=0pt,%                                 Corners not rounded.
    boxrule=0pt,%                             No boxrule.
    colframe=white]{}{}%                      Make rest of the boxrule invisible.
\ifstrempty{#1}{%                         If you didn't specify the optional argument of Theorem ...
    \begin{int_example}%                     ... then open a normal Theorem ...
}{%                                       ... else ...
    \begin{int_example}[#1]%                 ... open a Theorem and use the optional argument.
        }%
        }{%
    \end{int_example}%                          Close every environment.
    \end{tcolorbox}%
}

%\usepackage{thmtools}
%\renewcommand{\listtheoremsname}{Sätze und Definitionen}


% Complex number stuff
\renewcommand{\Re}{\text{Re}}
\renewcommand{\Im}{\text{Im}}
\def\i{\mathscr{i}}
\def\e{e}

% Various sets
\def\R{\ensuremath{\mathbb{R}}}
\def\C{\ensuremath{\mathbb{C}}}
\def\N{\ensuremath{\mathbb{N}}}
\def\Z{\ensuremath{\mathbb{Z}}}
\def\H{\ensuremath{\mathbb{H}}}
\def\K{\ensuremath{\mathbb{K}}}
\def\Rnn{\ensuremath{\mathbb{R}_{\geq 0}}}
\def\Part{\operatorname{Part}}
\def\OPart{\operatorname{OPart}}
\def\Seg{\operatorname{Seg}}

% Topology stuff
\def\Interior#1{\overset{\circ}{#1}}
\def\Closure#1{\overline{#1}}

% Defining equivalence
\def\ShortDefIff{:\Leftrightarrow}
\def\DefIff{:\Longleftrightarrow}

% smaller matrix env 
\newenvironment{smallpmatrix}
{\left(\begin{smallmatrix}}
        {\end{smallmatrix}\right)}
% Calculus
\def\d{\ensuremath{\mathrm{d}}}
\def\D{\ensuremath{\mathrm{D}}}

% Linear algebra
\def\Norm#1{\lVert#1\rVert}

% kontextsensitives Spatium. vgl. http://constantinfreitag.de/hp_docs/Freitag_MyP_2015_LaTeX-Einfuehrung_fuer_Linguisten.pdf
\usepackage{xspace}

% Abbreviations
\def\zB{z.\,B.\xspace}
\def\iA{i.\,A.\xspace}
\def\ua{u.\,a.\xspace}
\def\idR{i.\,d.\,R.\xspace}
\def\st{s.\,t.\xspace}
\def\resp{resp.\xspace}
\def\wrt{w.\,r.\,t.\xspace}
\def\etc{etc.\xspace}

% the following are just some defs for maths stuff - can freely be removed / modified

% optimization 
\def\argmin{\operatornamewithlimits{arg\, min}}
\def\argmax{\operatornamewithlimits{arg\, max}}
\def\CV{\operatorname{CV}}

% Landau symbols
% TODO: rename to BigO, SmallO to have \O for orthogonal group
\def\O{\mathcal{O}}
\def\o{\mathcal{o}}

% some matrix groups
\def\Ortho{\mathbf{O}}
\def\GL{\mathbf{GL}}
\def\SO{\mathbf{SO}}

% specialized symbols
\def\PcwPoly{\mathrm{PcwPoly}}
\def\Poly{\mathrm{Poly}}
\def\deg{\operatorname{deg}}
\def\dof{\operatorname{dof}}
\def\sgn{\operatorname{sgn}}

% Statistics
\def\SE{\operatorname{SE}}
\def\Var{\operatorname{Var}}

% Programming stuff
% C++
\def\CC{{C\nolinebreak[4]\hspace{-.05em}\raisebox{.4ex}{\tiny\bf ++}}\xspace}
% C#
\def\CSharp{{C\nolinebreak[4]\hspace{-.05em}\raisebox{.4ex}{\tiny\bf \#}}\xspace}
% F#
\def\FSharp{{F\nolinebreak[4]\hspace{-.05em}\raisebox{.4ex}{\tiny\bf \#}}\xspace}

% Units of measure
\def\MiB{\,\mathrm{MiB}}
\def\KiB{\,\mathrm{KiB}}
\def\s{\,\mathrm{s}}
\def\GHz{\,\mathrm{GHz}}
\def\MHz{\,\mathrm{MHz}}
\def\kHz{\,\mathrm{kHz}}

% for dummy text
\usepackage{lipsum}
