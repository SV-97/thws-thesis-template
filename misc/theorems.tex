
% custom theorem style für newlines zu beginn des Satzes
\newtheoremstyle{mainTheoremStyle}%    <name>
{\topsep}%   <space above>
{\topsep}%   <space below>
{}%  <body font> % was \itshape
{}%          <indent amount>
{\bfseries}% <Theorem head font>
{.}%         <punctuation after theorem head>
{\newline}%  <space after theorem head> (default .5em)
{}%          <Theorem head spec>
\theoremstyle{mainTheoremStyle}

% internal / raw theorem environments
\newtheorem{int_theorem}{Theorem}[section]
\newtheorem{int_definition}[int_theorem]{Definition}
\newtheorem{notation}[int_theorem]{Notation}
\newtheorem{int_lemma}[int_theorem]{Lemma}
\newtheorem{int_corollary}[int_theorem]{Corolloary}
\newtheorem{int_remark}[int_theorem]{Remark}
\newtheorem{int_example}[int_theorem]{Example}

\newenvironment{theorem}[1][]{%        % Create new environment which wraps our Theorem into a tcolorbox.
\begin{tcolorbox}[
    breakable,
    colback=blue!5!white,%     Background color.
    width=\dimexpr\linewidth+10pt\relax,%     Allow your box to be bigger than \linewidth ...
    enlarge left by=-5pt,%                    ... in order to have the text properly aligned. ...
    enlarge right by=-5pt,%                   ... Note that boxsep = -enlargeLeft = -enlargeRight = 0.5*enlargement of width. ...
    boxsep=5pt,%                              ... This is necessary to keep everything good looking.
    left=0pt,%                                Avoid extra space on the left, ...
    right=0pt,%                               ... right, ...
    top=0pt,%                                 ... top, ...
    bottom=0pt,%                              ... and bottom.
    arc=0pt,%                                 Corners not rounded.
    boxrule=0pt,%                             No boxrule.
    colframe=white]{}{}%                      Make rest of the boxrule invisible.
\ifstrempty{#1}{%                         If you didn't specify the optional argument of Theorem ...
    \begin{int_theorem}%                     ... then open a normal Theorem ...

}{%                                       ... else ...
    \begin{int_theorem}[#1]%                 ... open a Theorem and use the optional argument.
        }%
        }{%
    \end{int_theorem}%                          Close every environment.
    \end{tcolorbox}%
}

\newenvironment{definition}[1][]{%        % Create new environment which wraps our Theorem into a tcolorbox.
\begin{tcolorbox}[
    breakable,
    colback=red!5!white,%     Background color.
    width=\dimexpr\linewidth+10pt\relax,%     Allow your box to be bigger than \linewidth ...
    enlarge left by=-5pt,%                    ... in order to have the text properly aligned. ...
    enlarge right by=-5pt,%                   ... Note that boxsep = -enlargeLeft = -enlargeRight = 0.5*enlargement of width. ...
    boxsep=5pt,%                              ... This is necessary to keep everything good looking.
    left=0pt,%                                Avoid extra space on the left, ...
    right=0pt,%                               ... right, ...
    top=0pt,%                                 ... top, ...
    bottom=0pt,%                              ... and bottom.
    arc=0pt,%                                 Corners not rounded.
    boxrule=0pt,%                             No boxrule.
    colframe=white]{}{}%                      Make rest of the boxrule invisible.
\ifstrempty{#1}{%                         If you didn't specify the optional argument of Theorem ...
    \begin{int_definition}%                     ... then open a normal Theorem ...
}{%                                       ... else ...
    \begin{int_definition}[#1]%                 ... open a Theorem and use the optional argument.
        }%
        }{%
    \end{int_definition}%                          Close every environment.
    \end{tcolorbox}%
}

\newenvironment{lemma}[1][]{%        % Create new environment which wraps our Theorem into a tcolorbox.
\begin{tcolorbox}[
    breakable,
    colback=yellow!5!white,%     Background color.
    width=\dimexpr\linewidth+10pt\relax,%     Allow your box to be bigger than \linewidth ...
    enlarge left by=-5pt,%                    ... in order to have the text properly aligned. ...
    enlarge right by=-5pt,%                   ... Note that boxsep = -enlargeLeft = -enlargeRight = 0.5*enlargement of width. ...
    boxsep=5pt,%                              ... This is necessary to keep everything good looking.
    left=0pt,%                                Avoid extra space on the left, ...
    right=0pt,%                               ... right, ...
    top=0pt,%                                 ... top, ...
    bottom=0pt,%                              ... and bottom.
    arc=0pt,%                                 Corners not rounded.
    boxrule=0pt,%                             No boxrule.
    colframe=white]{}{}%                      Make rest of the boxrule invisible.
\ifstrempty{#1}{%                         If you didn't specify the optional argument of Theorem ...
    \begin{int_lemma}%                     ... then open a normal Theorem ...
}{%                                       ... else ...
    \begin{int_lemma}[#1]%                 ... open a Theorem and use the optional argument.
        }%
        }{%
    \end{int_lemma}%                          Close every environment.
    \end{tcolorbox}%
}

\newenvironment{corollary}[1][]{%        % Create new environment which wraps our Theorem into a tcolorbox.
\begin{tcolorbox}[
    breakable,
    colback=purple!5!white,%     Background color.
    width=\dimexpr\linewidth+10pt\relax,%     Allow your box to be bigger than \linewidth ...
    enlarge left by=-5pt,%                    ... in order to have the text properly aligned. ...
    enlarge right by=-5pt,%                   ... Note that boxsep = -enlargeLeft = -enlargeRight = 0.5*enlargement of width. ...
    boxsep=5pt,%                              ... This is necessary to keep everything good looking.
    left=0pt,%                                Avoid extra space on the left, ...
    right=0pt,%                               ... right, ...
    top=0pt,%                                 ... top, ...
    bottom=0pt,%                              ... and bottom.
    arc=0pt,%                                 Corners not rounded.
    boxrule=0pt,%                             No boxrule.
    colframe=white]{}{}%                      Make rest of the boxrule invisible.
\ifstrempty{#1}{%                         If you didn't specify the optional argument of Theorem ...
    \begin{int_corollary}%                     ... then open a normal Theorem ...
}{%                                       ... else ...
    \begin{int_corollary}[#1]%                 ... open a Theorem and use the optional argument.
        }%
        }{%
    \end{int_corollary}%                          Close every environment.
    \end{tcolorbox}%
}

\newenvironment{remark}[1][]{%        % Create new environment which wraps our Theorem into a tcolorbox.
\begin{tcolorbox}[
    breakable,
    colback=green!5!white,%     Background color.
    width=\dimexpr\linewidth+10pt\relax,%     Allow your box to be bigger than \linewidth ...
    enlarge left by=-5pt,%                    ... in order to have the text properly aligned. ...
    enlarge right by=-5pt,%                   ... Note that boxsep = -enlargeLeft = -enlargeRight = 0.5*enlargement of width. ...
    boxsep=5pt,%                              ... This is necessary to keep everything good looking.
    left=0pt,%                                Avoid extra space on the left, ...
    right=0pt,%                               ... right, ...
    top=0pt,%                                 ... top, ...
    bottom=0pt,%                              ... and bottom.
    arc=0pt,%                                 Corners not rounded.
    boxrule=0pt,%                             No boxrule.
    colframe=white]{}{}%                      Make rest of the boxrule invisible.
\ifstrempty{#1}{%                         If you didn't specify the optional argument of Theorem ...
    \begin{int_remark}%                     ... then open a normal Theorem ...
}{%                                       ... else ...
    \begin{int_remark}[#1]%                 ... open a Theorem and use the optional argument.
        }%
        }{%
    \end{int_remark}%                          Close every environment.
    \end{tcolorbox}%
}

\newenvironment{example}[1][]{%        % Create new environment which wraps our Theorem into a tcolorbox.
\begin{tcolorbox}[
    breakable,
    colback=yellow!5!white,%     Background color.
    width=\dimexpr\linewidth+10pt\relax,%     Allow your box to be bigger than \linewidth ...
    enlarge left by=-5pt,%                    ... in order to have the text properly aligned. ...
    enlarge right by=-5pt,%                   ... Note that boxsep = -enlargeLeft = -enlargeRight = 0.5*enlargement of width. ...
    boxsep=5pt,%                              ... This is necessary to keep everything good looking.
    left=0pt,%                                Avoid extra space on the left, ...
    right=0pt,%                               ... right, ...
    top=0pt,%                                 ... top, ...
    bottom=0pt,%                              ... and bottom.
    arc=0pt,%                                 Corners not rounded.
    boxrule=0pt,%                             No boxrule.
    colframe=white]{}{}%                      Make rest of the boxrule invisible.
\ifstrempty{#1}{%                         If you didn't specify the optional argument of Theorem ...
    \begin{int_example}%                     ... then open a normal Theorem ...
}{%                                       ... else ...
    \begin{int_example}[#1]%                 ... open a Theorem and use the optional argument.
        }%
        }{%
    \end{int_example}%                          Close every environment.
    \end{tcolorbox}%
}

%\usepackage{thmtools}
%\renewcommand{\listtheoremsname}{Sätze und Definitionen}
